\message{ !name(turning_sect.tex)}
\message{ !name(turning_sect.tex) !offset(-2) }
\section{Turning}
The robot's turning algorithm takes advantage of the fact that the
course only contains turns of approximately 90 degrees %degrees or deg.?
The goal of the turning algorithm is not to pivot exactly 90 degrees
and then move in a straight line every time, but rather to quickly execute a precise turn which results
in the robot being approximately parallel to and 20 cm away from the
opposite wall, after which the robot can easily continue using sonar
data to make accurate adjustments.

\subsection{Detecting a Turn}

\subsection{Using the Rotary Encoders}

The optical rotary encoders on each wheel of the Amigobot provide an
incredibly fine yet robust way to detect the rotational position of
each wheel. Each wheel's position can be loaded through \verb+SCOMP+'s
I/O through the I/O addresses \verb+0x80, 0x81, 0x88,+ and
\verb+0x89+, which correspond respectively to the given address names
\verb+LPOSLOW, LPOSHIGH, RPOSLOW,+ and \verb+RPOSHIGH+. The position
datum from each encoder is a 32-bit number. For I/O purposes, this datum
is split into two 16-bit numbers (the upper 16 bits and the lower 16
bits), each of which corresponds to the ``low'' or ``high'' I/O
address for each wheel's position.

\subsubsection{Physical Characteristics the Rotary Encoders}

The encoder datum increments by \(39000\) for each revolution of the
wheel. The left encoder increments when the left wheel is in forward
motion, while the right encoder decrements when the right wheel is in
forward motion. Each wheel has a diameter of 10 cm, which results in a
path of 31.42 cm being traversed for each wheel revolution so long as
traction is mantained. Since there are \(39000\) ``ticks'' per
revolution, one centimeter of linear wheel motion corresponds to
\(1241.41\) ticks. This results in very large-valued encoder data for
relatively short distances. In order to simplify calculations and
prevent bit carries between two 16-bit numbers (the high and low
data), it is best to perform calculations on just one 16-bit number
which can be produced by combining the upper 8 bits of the ``low''
datum with the  lower 8 bits of the ``high'' datum, resulting in a
reduction in encoder resolution by a factor of 256.  After shifting and truncating the two
16-bit numbers into one 16-bit number, the physical characteristics
of the encoder transform as such: 
\[
152.34 \frac{\text{ticks}}{\text{wheel revolution}}\\
4.85 \frac{\text{ticks}}{\text{cm of linear wheel motion}}
\]


\message{ !name(turning_sect.tex) !offset(-51) }

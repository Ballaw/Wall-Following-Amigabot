\section{Executive Summary}

An SCOMP wall following program was designed for the Amigobot
using sonar feedback to determine what velocity commands to send to
each wheel. This program enables the robot to locate and travel
parallel to a wall with inner and outer corners of \(90^\circ\). The
solution to this challenge is a state machine program that executes
commands based on the current state and the measured distance to the
nearest wall. The state machine consists of four states: ``forward
motion,'' ``inside turn,'' ``adjust outward,'' and ``adjust inward.''
In ``forward motion,'' the robot moves forward full speed until it senses a
wall in front of it or no wall is sensed. If a wall is sensed in front
of the Amigobot, the program switches to the ``inside turn'' state,
turns \(90^\circ\), and then proceeds to travel forward. If no wall is
sensed, the Amigobot will adjust inward until it is parallel to the
wall, and then continue on a straight path. A switch is used to toggle
between following a wall on the left side or on the right side of the
robot. The switch position determines turning direction and selects
the ultrasonic sensors used to measure distance to the wall. Parallel motion to
the wall is maintained by switching to ``adjust outward'' and ``adjust
inward'' states, which correct the robot trajectory if the measured
distance is not within an acceptable range of \(20\pm1\) cm. Inputs to
the state machine are ultrasonic sensory data. The outputs are
velocities of the left and right
wheels. During the final demonstration, the robot running this program
was moderately successful in following the test course. It
successfully navigated the wall on the right side and received an
accuracy bonus. While the robot completed the course when following
the wall on its left side, it did not receive an accuracy bonus and
did not finish the course in the desired time frame.



\section{Conclusions}

\subsection{Overview}

Overall, the design allowed the robot to successfully complete the
course. The design met all specifications. However, maintaining a
constant distance to the wall proved to be a challenge. While the
robot did not have problems maintaining accuracy when following the
right wall, it was unable to produce the same results when following the
left wall. Accuracy on the right wall following depended strongly on
suppressing the speed of the robot. The robot completed both courses
without collisions and penalties. 

\subsection{Strengths and Weaknesses of Design}

There are several strengths of this design. The robot is able to
successfully complete the course and avoid collisions. The design
allows the robot to run at variable speeds chosen by the user by
selecting switches on the DE2 board. The informative display greatly
helps in debugging processes and provides insight into how the program
is running. 

A weakness of this design is that the Amigobot cannot utilize its
speediness and complete the course successfully without
collisions. Extra time, in the form of slow movement, must be used in
order to accurately follow the wall. Additionally, the algorithm
itself is inefficient and produces inconsistent results. The
inefficiency stems from the multiple 100  ms wait loops in the adjustment states that
slow down sensory input tremendously. Wall following inconsistency may be attributed to
an inflexible algorithm unable to handle sensor inaccuracies.

\subsection{Recommendations}

For future work, a useful development would be to show the moving
states on the LCD display as initially proposed. This would allow for quicker and more
effective debugging and would be a useful addition to the display. 

Another development would be to encode the green LEDs to show more
functions as they are executed. The current design uses the green LEDs
only as a completion bar for the ``inside turn'' state. Animating the LEDs
for other states could aid in debugging purposes and improve the
display. 

While this design takes the minimum of two adjacent sensory values, it
could be edited to use weighted values of the sensors. Allowing some
sensors to take precedence over others would allow the robot to
accurately determine its orientation relative to the wall. 

Due to limited computational resources in assembly language, the
measurement algorithm used to detect turns is not as accurate as
desired. A major improvement to the program would be to compute a more
accurate measurement algorithm, such as averaging the forward sensor
data over multiple instances of time.

\subsection{Optimizations}

In order to optimize the design, the 100 ms wait loops should be removed
from the adjustment states. The loops decrease the frequency of adjustment
greatly, which prevents the robot from moving too quickly. In order to
remove the wait loops, the adjustment states would have to be
rewritten in order to remove cumulative additions. 

Additionally, the invalid value of the sonar sensors should be changed
from \verb+0xFFFF+ (\(-1\)) to a large positive number,
\verb+0x7FFF+ (32767). Since the code implemented includes many \verb+JNEG+
instructions, this would prevent having to include special
consideration for invalid sensor readings. Ideally, this would be
changed at the hardware level of the sonar sensors. 
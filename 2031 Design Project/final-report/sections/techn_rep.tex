\section{Technical Results}

\subsection{Final Demonstration Results}

The course used for the final demonstration consisted of five straight
wall segments, with four outside turns and four inside turns. Each
straight segment of the course was approximately four feet. 

The results of the final demonstration showing the \textbf{expected},
\textbf{actual}, and \textbf{difference}  in values of scores can be found in
Table~\ref{results}. The expected values for the completion, accuracy,
and time are based on test runs performed during lab periods. 

\begin{table}[h!]
  \centering \cprotect \fbox{
    \begin{minipage}{5.5in}
      \centering
      \small
     \cprotect \caption{Results of wall following demo}
      \label{results}
      \begin{tabular}{|c|c|c|c|c|c|c|c|c|c|}
        \hline
        &\multicolumn{3}{|c|}{\textbf{Completeness} (\%)}&
        \multicolumn{3}{|c|}{\textbf{Accuracy} (\%)}&
        \multicolumn{3}{|c|}{\textbf{Time} (min:sec)} \\ \hline
        &\textbf{Exp.}&\textbf{Act.}&\textbf{Diff.}&
        \textbf{Exp.}&\textbf{Act.}&\textbf{Diff.}&
        \textbf{Exp.}&\textbf{Act.}&\textbf{Diff.}  \\
        \hline
        \textbf{Right Wall} & 100\% & 100\% & - & 100\% & 100\% & - &
        1:15 & 1:23 & 0:08 \\ \hline
        \textbf{Left Wall} & 100\% & 100\% & - & 100\% & 0\% & 100\% &
        1:15 & 5:52 & 4:37 \\ \hline
        \multicolumn{7}{|l|}{\textbf{Total Time}} & 2:30 & 6:15 & 3:45 \\ \hline
      \end{tabular}
    \end{minipage}
  }
\end{table} 

The robot did not collide with the wall on either run. It received an
accuracy bonus for five out of the ten sections it completed. 

The robot completed the course following the right wall in one minute
and 23 seconds, which was eight seconds slower than the expected
time. The robot was 100\% accurate during this run.

The team experienced problems during the left wall following
demonstration. The problems encountered were due to the prevalence of
sequential inside turns in the particular left-wall following course
presented. One issue identified was that the robot was inconsistent in
detecting its distance to a forward wall, which is crucial to execute
an inside turn. The result was that the robot would frequently turn
too early or too late and would be unable to readjust to recover in
time for the next turn. The
problem was in the range-calculating algorithm implemented and not in the
hard-coded \(90^\circ\) turns using rotary encoder measurements, since
the turns executed were exceptionally consistent. The time to complete the
course was a result of several trial runs due to near collisions. The
robot was switched for another robot at the behest of the professor
due to strange sensor behavior on the left side. This event caused
additional setup and upload delays. The robot was able to succesfully
complete the course, but did so without maintaining a constant
distance to the wall.

\subsection{Outcomes}

The results from the final demonstration and test runs suggest that
the robot can consistently and accurately complete a wall following
course predominantly composed of outside turns. When performing
sequential inside turns, the robot is often unable to successfully
navigate the course. In either case, the robot must travel very slowly
to avoid collisions with the wall. 







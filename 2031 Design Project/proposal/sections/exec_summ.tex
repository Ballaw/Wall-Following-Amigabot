\section{Executive Summary}
The team will design a SCOMP wall following program for the Amigobot
using sonar feedback to determine what velocity commands to send to
each wheel, enabling the robot to locate and travel parallel to a
wall. The wall has inner and outer corners of approximately \(90^\circ\) that the robot must
navigate. 

 The team will design a solution to this problem in the form
of a state machine program that executes commands based on the current
state and the measured distance to the nearest wall. The state machine
consists of six states: ``forward motion,'' ``inside turn,'' ``outside
turn,'' ``adjust outward,'' and ``adjust inward.'' In ``forward
motion,'' the robot will move forward full speed until it senses a
wall in front of it or no wall is sensed. If a wall is sensed in front
of the Amigobot, the program switches to the ``inside turn'' state,
turns \(90^\circ\), then proceeds to travel forward. If no wall is sensed, the
Amigobot will switch to the ``outside turn'' state, turn \(90^\circ\), and
continue forward. A switch is used to toggle between following a wall
on the left side or on the right side of the robot. The switch position will determine
turning direction, and it will select which ultrasonic sensors will be
used to measure distance. Parallel motion to the wall will be
maintained by switching to ``adjust outward'' and ``adjust inward''
states, which will correct the robot trajectory if the measured
distance is not within an acceptable range of \(20\pm2\) cm.  

The strength of this approach is that navigation of wall corners will not
depend on sensor data. Sensor data is not a dependable measure of
distance to the wall when the sensor is not perpendicular to the
wall. As the Amigobot turns, the sensors are not aligned with the wall
and give incorrect measurements of distance, which can result in
collisions with the wall and loss of orientation relative to the wall.
